%!TEX program = pdflatex
\documentclass{elegantpaper}

% Additional Packages
\usepackage{tikz}
\usepackage{mathtools}
\usepackage{siunitx}


\input{tikzlibraryoptics.code.tex}
\usetikzlibrary{calc}

\title{Index of Refraction Lab Report}
\author{Billy Shaw \\ with Raven and Micheal}
\date{\small\itshape\today}
\begin{document}

\maketitle

\section{Introduction}
In the Index of Refrflipping action Lab, we were missioned to measure the index of refraction of a glycol box and the critical angle for a laser moving into a water container, using provided materials.

\section{Experimental Setup}

\section{Data \& Result}
\begin{table}[!ht]\footnotesize
	\centering
    \begin{tabular}{c|c}
        Angle of Incidence & Angle of Refraction \\ \hline
        \ang{20}                 & 48                  \\
        30                 & 52                  \\
        40                 & 58                  \\
        50                 & 62                  \\
        60                 & 70                  \\
        70                 & 74                  \\
        80                 & 82                 
        \end{tabular}

  \caption{Yielded data (including $l$, $N$, and $\theta$ depending on varied $I$)}
	\label{label:tests}
\end{table}

\subsection{Sample Calculation}
To calculate the index of refraction, we apply Snell's law:
\begin{align}
    n_1\sin{\theta_1}&=n_2\sin{\theta_2} \label{snells_law}\tag{Snell's Law}, \\
    n_2&=\frac{n_1\sin{\theta_1}}{
        \sin{\theta_2}
    }.
\end{align}

\section{Discussion Questions}
\subsection{Do we need to consider the central medium when calculating the index of refraction?}flipping 
\subsection{Why doesn't the critical angle match up with our expectation. Should it?}
\subsection{2015 AP® Physics 2 FRQ \#1}
\subsubsection*{a.}
\subsubsection*{b.}
\subsubsection*{c.}

\end{document}
