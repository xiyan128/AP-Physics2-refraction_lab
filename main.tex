%!TEX program = pdflatex
\documentclass{elegantpaper}

% Additional Packages
\usepackage{tikz}
\usepackage{mathtools}
\usepackage{siunitx}
\usepackage{fancyhdr}
\usepackage{graphicx, wrapfig, subcaption, setspace, booktabs}
\usepackage{lastpage}
\usepackage[myheadings]{fullpage}
\usepackage{mathtools}


\newcommand{\HRule}[1]{\rule{\linewidth}{#1}}

\input{tikzlibraryoptics.code.tex}
\usetikzlibrary{calc}

\onehalfspacing
\setcounter{tocdepth}{0}
\setcounter{secnumdepth}{0}



%-------------------------------------------------------------------------------
% HEADER & FOOTER
%-------------------------------------------------------------------------------
\pagestyle{fancy}
\fancyhf{}
\setlength\headheight{15pt}
\fancyhead[L]{Billy Shaw}
\fancyhead[R]{Index of Refraction Lab Report}
\fancyfoot[R]{Page \thepage\ of \pageref{LastPage}}

%-------------------------------------------------------------------------------
% TITLE PAGE
%-------------------------------------------------------------------------------

\begin{document}

\title{ \normalsize \textsc{AP Physics 2}
		\\ [2.0cm]
		\HRule{0.5pt} \\
		\LARGE \textbf{\uppercase{Index of Refraction Lab Report}}
		\HRule{2pt} \\ [0.5cm]
		\normalsize \today \vspace*{5\baselineskip}}

\date{}

\author{
		Billy \\ 
		with Raven \& Michael}

\maketitle

\clearpage

\section{Introduction}
In the Index of Refraction Lab, we were missioned to measure the index of refraction of a glycol box, as well as the critical angle for a laser moving into a water container, using provided materials.

\section{Experimental Setup}
\begin{figure}[!ht]
	\centering
	\includegraphics[width=12cm]{setup-1.pdf}
	\caption{Experiment Setup 1 (reflection not drawn); in our expectation, $n_1\sin{\theta_1}=n_2\sin{\theta_2}$}
	\centering
  \label{label:Setup}
\end{figure}

\begin{figure}[!ht]
	\centering
	\includegraphics[width=12cm]{setup-2.pdf}
	\caption{Experiment Setup 2}
	\centering
  \label{label:Setup}
\end{figure}
\clearpage

\section{Data \& Result}
\begin{table}[!ht]\footnotesize
	\centering
    \begin{tabular}{c|c}
        \multicolumn{2}{c}{\textbf{Yielded Data for Experiment 1}}\\ \hline
        Angle of Incidence & Angle of Refraction \\ \hline
        \ang{70}&\ang{42}\\
        \ang{60}&\ang{38}\\
        \ang{50}&\ang{32}\\
        \ang{40}&\ang{28}\\
        \ang{30}&\ang{20}\\
        \ang{20}&\ang{16}\\
        \ang{10}&\ang{8}\\
        \hline          
        \end{tabular}

  \caption{Yielded data for experiment 1, including Angles of Refraction ($\theta_2$) and Angles of Incidence ($\theta_1$)}
	\label{table: index of refraction}
\end{table}

\begin{table}[!ht]\footnotesize
	\centering
    \begin{tabular}{c|c}
        \multicolumn{2}{c}{\textbf{Yielded Data for Experiment 2}}\\ \hline
        Critical Angle & \ang{51} \\ \hline
        $n$ of the Container & about 1.49 \\ \hline
        \end{tabular}

  \caption{Yielded data for experiment 2, where $n$ is the refractive index of the container}
	\label{table: experiment 2}
\end{table}

\subsection{Sample Calculation}
To calculate the index of refraction, we apply Snell's law:
\begin{align}
    n_1\sin{\theta_1}&=n_2\sin{\theta_2} \label{snells_law}\tag{Snell's Law}, \\
    n_2&=\frac{n_1\sin{\theta_1}}{
        \sin{\theta_2}
    }.
\end{align}
Take row $\theta_1=\ang{50}, \theta_2=\ang{32}$ as an example:
\begin{align*}
    n_2&=\frac{n_1\sin{\theta_1}}{\sin{\theta_2}}=\frac{1\times\sin{\ang{50}}}{\sin{\ang{32}}}\approx1.45
\end{align*}
Calculating critical angle based on known index of refraction:
\begin{align*}
  n_1\sin{\theta_c}&=n_2\sin{\theta_2} \label{snells_law}\tag{Snell's Law} \\
  \sin \theta_{c}&=\frac{n_{2}}{n_{1}} \sin{\ang{90}}\\
  \theta_{c}&=\arcsin{\frac{n_{2}}{n_{1}}} \label{Critical Angle Equation} \tag{Critical Angle Equation}
\end{align*}
\clearpage
\subsection{Result}
\begin{table}[!ht]\footnotesize
	\centering
    \begin{tabular}{c|c|c}
        \multicolumn{3}{c}{\textbf{Result for Experiment 1}}\\ \hline
        Angle of Incidence & Angle of Refraction & Reflective Index \\ \hline
        \ang{70}&\ang{42}&1.40\\
        \ang{60}&\ang{38}&1.41\\
        \ang{50}&\ang{32}&1.45\\
        \ang{40}&\ang{28}&1.37\\
        \ang{30}&\ang{20}&1.46\\
        \ang{20}&\ang{16}&1.24*\\
        \ang{10}&\ang{8}&1.25*\\
        \hline
        \multicolumn{1}{c}{\textbf{Average $n$}} & \multicolumn{1}{c}{} & 1.42\\
        \hline
        \end{tabular}

  \caption{Result for experiment 1, derived by using Equation (1); the average reflective index for glycol, $n\approx1.42$ (*since the last two row appear to be outliers, they are ignored when calculating the average value).}
	\label{table: index of refraction}
\end{table}


\section{Discussion Questions}
\subsection{Do we need to consider the central medium when calculating the index of refraction?}
\begin{figure}[!ht]
	\centering
	\includegraphics[width=12cm]{disc-1.pdf}
	\caption{Ray Diagram for Experiment 1}
	\centering
  \label{label:Disc-1}
\end{figure}
No, we don't need to consider the central when calculating the reflective index. The reason is that since $l_1 \parallel l_2$, $\theta_{g1}=\theta_{g2}$. By employing Snell's law,
\begin{equation*}
  \begin{cases}
    n_1\sin{\theta_1}=n_g\sin{\theta_{g1}}\\
    n_g\sin{\theta_{g2}}=n_2\sin{\theta_1}\\
    \theta_{g1}=\theta_{g2}
  \end{cases},
\end{equation*}
we get $n_1\sin{\theta_1}=n_2\sin{\theta_2}$, which doesn't involve $n_g$. Thus, it should be fine to ignore the intermedium.
\subsection{Why doesn't the critical angle match up with our expectation. Should it?}
The critical angle of air to water, according to Snell's law, should be the $\theta_1$ when $\theta_2=\ang{90}$, $n_1=1$, and $n_2=1.33$. Plugging in the numbers into the critical angle equation we derived before, we end up with the expected critical angle: $\theta_{expected}\approx\ang{48.75}$.

Though the measured critical angle (\ang{51}) is not far away from the expected one, we would admit that they still don't match. One possible explanation could be that our measurement is not accurate (surely our fault, because the equipment is said to be expansive). Another reason could be that the milky wather itself has a larger index of refraction (which, according to the equation, should lead to a larger critical angle).

Despite these intervening factors, the critical angle should match up with the calculated expectation -- that is, the container would not affect the result, as long as its refractive index is greater than that of the milky water (which we have proven to be true, as shown in the previous section). The reason is that since the critical angle $\theta_{c\prime}$ between the central meduim is $\arcsin{\frac{n_2}{n_i}}$ (where $n_i$ is the refractive index of the intermedium), by applying Snell's on the air-container surface, we get $n_1\sin{\theta_c}=n_i\sin{\theta_{c\prime}}$, which could be rewriten (by substituting $\theta_{c\prime}$ with $\frac{n_2}{n_i}$), as $\sin{\theta_c}=\frac{n_2}{n_1}$. The derived equation apparently has nothing to do with the container, yet being exactly the critical angle we expected. Thus, we should say that the expected and the experimental should match.
\clearpage
\subsection{2015 AP® Physics 2 FRQ \#1}
\subsubsection*{a.}
\begin{figure}[!ht]
	\centering
	\includegraphics[width=12cm]{figure-1.pdf}
	\caption{Ray Diagram for Part A}
	\centering
  \label{label:f1}
\end{figure}
\subsubsection*{b.}
\begin{figure}[!ht]
	\centering
	\includegraphics[width=12cm]{figure-2.pdf}
	\caption{Ray Diagram for Part B, where due to total internal reflection (no more energy loss on Air-Liquid surface), spot Y becomes brighter.}
	\centering
  \label{label:f1}
\end{figure}
\clearpage
\subsubsection*{c.}
\begin{figure}[!ht]
	\centering
	\includegraphics[width=12cm]{figure-3.pdf}
	\caption{Ray Diagram for Part C}
	\centering
  \label{label:f1}
\end{figure}
When the incident angle increases, spot that disappears, if any, would be the farther one (Y). It is because if and only if $\theta_3$ reaches the liquid-glass critical angle would no refracted beam pass incritical angle containerto the liquid to cause the second reflection (which forms spot Y).
\end{document}
